\documentclass[journal,12pt,twocolumn]{IEEEtran}

\usepackage{setspace}
\usepackage{gensymb}
\singlespacing
\usepackage[cmex10]{amsmath}

\usepackage{amsthm}

\usepackage{mathrsfs}
\usepackage{txfonts}
\usepackage{stfloats}
\usepackage{bm}
\usepackage{cite}
\usepackage{cases}
\usepackage{subfig}

\usepackage{longtable}
\usepackage{multirow}
\usepackage{adjustbox}
\usepackage{enumitem}
\usepackage{mathtools}
\usepackage{steinmetz}
\usepackage{tikz}
\usepackage{circuitikz}
\usepackage{verbatim}
\usepackage{tfrupee}
\usepackage[breaklinks=true]{hyperref}
\usepackage{graphicx}
\usepackage{tkz-euclide}
\usepackage{tikz}
\usetikzlibrary{shapes.multipart}

\usetikzlibrary{calc,math}
\usepackage{listings}
    \usepackage{color}                                            %%
    \usepackage{array}                                            %%
    \usepackage{longtable}                                        %%
    \usepackage{calc}                                             %%
    \usepackage{multirow}                                         %%
    \usepackage{hhline}                                           %%
    \usepackage{ifthen}                                           %%
    \usepackage{lscape}     
\usepackage{multicol}
\usepackage{chngcntr}

\DeclareMathOperator*{\Res}{Res}

\renewcommand\thesection{\arabic{section}}
\renewcommand\thesubsection{\thesection.\arabic{subsection}}
\renewcommand\thesubsubsection{\thesubsection.\arabic{subsubsection}}
% \renewcommand{\thefigure}{\theenumi}
\renewcommand\thesectiondis{\arabic{section}}
\renewcommand\thesubsectiondis{\thesectiondis.\arabic{subsection}}
\renewcommand\thesubsubsectiondis{\thesubsectiondis.\arabic{subsubsection}}
\usepackage{tikz}
\usetikzlibrary{shapes,arrows}

\hyphenation{op-tical net-works semi-conduc-tor}
\def\inputGnumericTable{}                                 %%

\lstset{
%language=C,
frame=single, 
breaklines=true,
columns=fullflexible
}
\begin{document}


\newtheorem{theorem}{Theorem}[section]
\newtheorem{problem}{Problem}
\newtheorem{proposition}{Proposition}[section]
\newtheorem{lemma}{Lemma}[section]
\newtheorem{corollary}[theorem]{Corollary}
\newtheorem{example}{Example}[section]
\newtheorem{definition}[problem]{Definition}

\newcommand{\BEQA}{\begin{eqnarray}}
\newcommand{\EEQA}{\end{eqnarray}}
\newcommand{\define}{\stackrel{\triangle}{=}}
\bibliographystyle{IEEEtran}
\raggedbottom
\setlength{\parindent}{0pt}
\providecommand{\mbf}{\mathbf}
\providecommand{\pr}[1]{\ensuremath{\Pr\left(#1\right)}}
\providecommand{\qfunc}[1]{\ensuremath{Q\left(#1\right)}}
\providecommand{\sbrak}[1]{\ensuremath{{}\left[#1\right]}}
\providecommand{\lsbrak}[1]{\ensuremath{{}\left[#1\right.}}
\providecommand{\rsbrak}[1]{\ensuremath{{}\left.#1\right]}}
\providecommand{\brak}[1]{\ensuremath{\left(#1\right)}}
\providecommand{\lbrak}[1]{\ensuremath{\left(#1\right.}}
\providecommand{\rbrak}[1]{\ensuremath{\left.#1\right)}}
\providecommand{\cbrak}[1]{\ensuremath{\left\{#1\right\}}}
\providecommand{\lcbrak}[1]{\ensuremath{\left\{#1\right.}}
\providecommand{\rcbrak}[1]{\ensuremath{\left.#1\right\}}}
\theoremstyle{remark}
\newtheorem{rem}{Remark}
\newcommand{\sgn}{\mathop{\mathrm{sgn}}}
% \providecommand{\abs}[1]{\left\vert#1\right\vert}
% \providecommand{\res}[1]{\Res\displaylimits_{#1}} 
% \providecommand{\norm}[1]{\left\lVert#1\right\rVert}
% %\providecommand{\norm}[1]{\lVert#1\rVert}
% \providecommand{\mtx}[1]{\mathbf{#1}}
% \providecommand{\mean}[1]{E\left[ #1 \right]}
\providecommand{\fourier}{\overset{\mathcal{F}}{ \rightleftharpoons}}
%\providecommand{\hilbert}{\overset{\mathcal{H}}{ \rightleftharpoons}}
\providecommand{\system}{\overset{\mathcal{H}}{ \longleftrightarrow}}
	%\newcommand{\solution}[2]{\textbf{Solution:}{#1}}
\newcommand{\solution}{\noindent \textbf{Solution: }}
\newcommand{\cosec}{\,\text{cosec}\,}
\providecommand{\dec}[2]{\ensuremath{\overset{#1}{\underset{#2}{\gtrless}}}}
\newcommand{\myvec}[1]{\ensuremath{\begin{pmatrix}#1\end{pmatrix}}}
\newcommand{\mydet}[1]{\ensuremath{\begin{vmatrix}#1\end{vmatrix}}}
\numberwithin{equation}{subsection}

\makeatletter
\@addtoreset{figure}{problem}
\makeatother
\let\StandardTheFigure\thefigure
\let\vec\mathbf
\renewcommand{\thefigure}{\theproblem}
\def\putbox#1#2#3{\makebox[0in][l]{\makebox[#1][l]{}\raisebox{\baselineskip}[0in][0in]{\raisebox{#2}[0in][0in]{#3}}}}
     \def\rightbox#1{\makebox[0in][r]{#1}}
     \def\centbox#1{\makebox[0in]{#1}}
     \def\topbox#1{\raisebox{-\baselineskip}[0in][0in]{#1}}
     \def\midbox#1{\raisebox{-0.5\baselineskip}[0in][0in]{#1}}
\vspace{3cm}
\title{Assignment 1}
\author{Ramneti Sai Karthik - EE18BTECH11037}
\maketitle
\newpage
\bigskip
\renewcommand{\thefigure}{\theenumi}
\renewcommand{\thetable}{\theenumi}
Download all latex-tikz codes from 
%
\begin{lstlisting}
https://github.com/KarthikRamneti/C-DS/blob/main/assignment1/assgn1.tex
\end{lstlisting}
\setcounter{figure}{0}
\section{Problem}
(Q 27) Consider the following C function.
\begin{lstlisting}
#include <stdio.h>
int r(){
    static int num = 7;
    return num--;
}

int main(){
    for (r();r();r())
        printf("%d",r());
    return 0;
} 
\end{lstlisting}
\setcounter{figure}{0}
Which of the following values will be displayed on execution of the programs?
\begin{enumerate}
    \item 41
    \item 52
    \item 63
    \item 630
\end{enumerate}

\section{Solution}
Answer : C) 52
\newline
\\
\textbf{Explanation:}
\newline
A static variable is a variable that persists its value across the various function calls. Static variables are variables that remain in memory while the program is running i.e. their lifetime is the entire program run.
\\
Here, \textbf{num} is the static variable and is initialized with the value = 7.
\\
\\
Execution Order of for loop is:

\tikzstyle{decision} = [diamond, draw, fill=white!20, 
    text width=4.5em, text badly centered, node distance=3cm, inner sep=0pt]
\tikzstyle{block} = [rectangle, draw, fill=white!20, 
    text width=7em, text centered, rounded corners, minimum height=1em]
\tikzstyle{line} = [draw, -latex']
\tikzstyle{cloud} = [draw, ellipse,fill=red!20, node distance=3cm,
    minimum height=2em]
\tikzstyle{sblock} = [rectangle, draw, fill=white!20, 
    text width=10em, text badly centered, rounded corners, minimum height=1em]


\begin{figure}[!h]
\centering
\begin{tikzpicture}[node distance = 2cm, auto]
    % Place nodes
    \node [cloud] (start) {start};
    \node [block, below of=start] (init1) {init statement};
    \node [decision, below of=init1] (decision1) { condition };
    \node [block, below of=decision1] (codeblock) {codeblock};
    \node [block, below of=codeblock] (increment) {increment statement};
    \node [cloud, below of=increment] (end) {end};
    
    % Draw edges
    \path [line] (start) -- (init1);
    \path [line] (init1) -- (decision1);
    \path [line] (decision1) -- ++(3,0) |-  node [near start] {false} (end);
    \path [line] (decision1) -- node [near start] {if condition true} (codeblock);
    \path [line] (codeblock) -- (increment);
    
    \path [line] (increment)  -- ++(-2,0) |-   (decision1);
\end{tikzpicture}
\caption{Execution Order of the for loop} \label{fig:M1}
\end{figure}


\section{Problem}
Here, if \textbf{r()} is called first time then the \textbf{"num"} static varible is initiated with 7 and \textbf{num-{}-} decrements the value of the \textbf{num} variable by 1 after returning the value \textbf{num} and if called next time, the \textbf{"num"} static varible is not initiated with 7.
\\
So, At the first r() call, num = 7 and r() returns 7, after the first call num value becomes 6 
and in the second call, num = 6, and r() return 6, after the first call num value becomes 5
\\
The execution of program is as follows and prints 52.


\begin{figure}[h]
\centering
\begin{tikzpicture}[node distance = 3cm, auto]
    % Place nodes
    \node [cloud] (start) {start};
    \node [sblock, below of=start] (init1) {r() call in init statement,num = 6, returns 7};
    \node [sblock, below of=init1] (decision1) {r() call in condition statement,num = 5, returns 6};
    \node [sblock, below of=decision1] (codeblock) {r() call in print statement,num = 4, returns 5 and \textbf{prints 5}};
    \node [sblock, below of=codeblock] (increment) {r() call in increment statement,num = 3, returns 4};
    
    \node [sblock, right of=decision1, node distance=6cm] (decision2) {r() call in condition statement,num = 2, returns 3};
    \node [sblock, below of=decision2] (codeblock2) {r() call in print statement,num = 1, returns 2 and \textbf{prints 2}};
    \node [sblock, below of=codeblock2] (increment2) {r() call in increment statement,num = 0, returns 1};
    
    \node [sblock, right of=decision2, node distance=6cm] (decision3) {r() call in condition statement,num = -1, returns 0};
    
    \node [cloud, below of=increment] (end) {end};
    
    % Draw edges
    \path [line] (start) -- (init1);
    \path [line] (init1) -- (decision1);
    \path [line] (decision3) -- ++(0,-3) |-  node [near start] {returned 0} (end);
    \path [line] (decision1) -- node [near start] {$6!=0$} (codeblock);
    \path [line] (codeblock) -- (increment);
    \path [line] (increment)  -- ++(3,0) |-   (decision2);
    
    \path [line] (decision2) -- node [near start] {$3!=0$} (codeblock2);
    \path [line] (codeblock2) -- (increment2);
    \path [line] (increment2)  -- ++(3,0) |-   (decision3);
    
\end{tikzpicture}
\caption{Program Execution Timeline} \label{fig:M1}
\end{figure}


\section{General Mathematical formula}
If the static variable num is initialised to a number of the form $\textbf{3k+1}$, where $k>=0$. Then the initialization call in for loop runs one time, the condition call, code block call and the increment call in the for loop runs k times because after this r() returns 0, so the condition statement will be false and the for loop terminates.Therefore the print statement is called k times. So it prints (num-2)(num-2-3)....2.
\\
\\
If the static variable num is initialised to a number which is not of the form $\textbf{3k+1}$, then the for loop runs indefinitely because the r() call in the
\\
\\
\\
\\
\\
\\
\\
\\
\\
\\
\\
\\
\\
\\
\\
\\
\\
\\
\\
\\
\\
\\
\\
\\
\\
\\
\\
\\
\\
\\
\\
\\
\\
\\
\\
\\
\\
condition statement will never return 0.
$
Output = \begin{cases} 
      (num-2)(num-5)....2 & num=3k+1 \\
      runs\,infinetly & else
   \end{cases}
$




\end{document}
